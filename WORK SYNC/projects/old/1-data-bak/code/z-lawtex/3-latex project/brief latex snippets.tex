

%Syntax for @startsection{name}{level}{indent}{abovespace}{belowspace}{format}
%The 4th argument is negative to suppress indenting after the section heading
\def\section{\@startsection{section}{2}{0pt}{-10pt}{2pt}{\centering\normalsize\bfseries}}%
\def\subsection{\@startsection{subsection}{2}{0pt}{-2pt}{2pt}{\hyphenpenalty=10000\normalsize\bfseries}}%
\def\subsubsection{\@startsection{subsubsection}{3}{\parindent}{-2pt}{2pt}{\hyphenpenalty=10000\bfseries}}%

%No numbers printed for section (remove \quad); I. Subsection; A. Subsubsection
\renewcommand{\thesection}{\hspace{-1em}}
\renewcommand{\thesubsection}{\Roman{subsection}}
\renewcommand{\thesubsubsection}{\Alph{subsubsection}.}

% \newcommand{\makecaption}{{

\singlespacing

\begin{centering}
\bf\scshape \@court \\~\\ 
\rm 

\begin{tabular}{p{.45\textwidth}|p{.45\textwidth}}
\cline{1-1}
{~

\raggedright\@firstparty, 

~

{\leftskip=1in plus 1.fill minus .5in%
\textit{\@firstpartytitle,}\par}

\hspace*{1in}v.

~ 

\raggedright\@secondparty, 

~

{\leftskip=1in plus 1.fill minus .5in%
\textit{\@secondpartytitle.}\par} 
~
} &
{\@actionnumber
} \\
\cline{1-1}
\end{tabular}

\end{centering}
}}

% end box
    
\newenvironment{rightbox}{
	%Start a new group to contain the def of the testbox and newline
	\bgroup
	%Store the box here
	\newbox\testbox
	%Change newline so that each line is an hbox
	\def\\{\unskip\egroup\hbox\bgroup\ignorespaces}
	%Start vbox{hbox{ ...
	\setbox\testbox=\vbox\bgroup\hbox\bgroup\ignorespaces}
	%End the last hbox and the vbox
	{\unskip\egroup\egroup
	%Output a fill, then the box
	\hskip 0pt plus 1.0fill \box\testbox \egroup }


